\documentclass{article}
\usepackage{graphicx} % Required for inserting images
\usepackage{listings}
\usepackage{xcolor}   % For adding colors to listings
\usepackage{amsmath}  % For math notation
\usepackage{fancyhdr}
\usepackage[T1]{fontenc}
\usepackage{titling}
\usepackage{verbatim} 
\usepackage{blindtext}
\usepackage{geometry}
\usepackage{array}

\geometry{
 a4paper,
 total={165mm,257mm},
% left=20mm,
 top=30mm,
 bottom=20mm
 }
% Configure listings
\lstset{
  language=Java,
  basicstyle=\ttfamily\small,
  keywordstyle=\color{blue},
  stringstyle=\color{red},
  commentstyle=\color{gray},
  numbers=left,
  numberstyle=\tiny\color{gray},
  stepnumber=1,
  frame=single,
  breaklines=true,
  captionpos=b,
  tabsize=2,
  showspaces=false,
  showstringspaces=false
}

\begin{document}

\pagestyle{fancy}

\fancyhf{} %clear existing header/footer
\renewcommand{\headrulewidth}{0pt}
\lhead{CS-3353-23334 Data Structures}
\rhead{Jimmy Nguyen}
\rfoot{\thepage}

\newenvironment{myindentpar}[1]%
 {\begin{list}{}%
         {\setlength{\leftmargin}{#1}}%
         \item[]%
 }
 {\end{list}}

\section*{Part 1: Big-O Notation}
For each of the following code snippets, determine the Big-O time complexity and explain your reasoning:
\subsection*{\underline {Code 1}}

\begin{lstlisting}
public void example1(int[] arr) 
{
  for (int i = 0; i < arr.length; i++) 
  {
    for (int j = i; j < arr.length; j++) 
    {
      System.out.println(arr[i] + arr[j]);
    } 
  }
}
\end{lstlisting}
\textbf{Solution:}
\begin{myindentpar}{.5cm}
The Big-O time complexity of Code 1 is \(O(n^2)\) where n is equal to arr.length. The outer for loop runs from i = 0 to i < arr.length, which means it iterates n times. The inner for loop runs from j = 1 to j < arr.length, which means it runs n times for every iteration of the outer loop. The 1 represents the line of code in the nested for loop that only iterates once. So the total number of iterations is n(n(1)) or \(n^2\).
\end{myindentpar}

\subsection*{\underline{Code 2}}

\begin{lstlisting}
public void example2(int[] arr) 
{
  for (int i = 0; i < arr.length; i++) 
  {
    if (arr[i] % 2 == 0) 
    {
      System.out.println(arr[i]);
    }
  }
}
\end{lstlisting}
\textbf{Solution:} 
\begin{myindentpar}{.5cm}
The Big-O time complexity of Code 2 is \(O(n)\) where n is the length of arr. The for loop iterates from 0 to i < arr.length, meaning that it runs n times, where n is arr.length. Inside the loop there is an if statement that operates for the duration of the loop. The print statement in the if statement also iterates for the duration of the loop. Since the loop runs n times, the time complexity is \(O(n)\)
\end{myindentpar}

\pagebreak

 \subsection*{\underline{Code 3}}

\begin{lstlisting}
public void example3(int n) 
{
  int i = 1;
  while (i < n) 
  {
    System.out.println(i);
    i *= 2;
  }
}
\end{lstlisting}
\textbf{Solution:} 
\begin{myindentpar}{.5cm}
The Big-O time complexity of Code 3 is O(log n). The while loop iterates where i starts at 1 and doubles itself until i is $\geq$ n. Because i doubles every iteration, after x iterations, i \( = 2^n \). Because the loop only runs while i < n, we need to solve \(2^x \geq n\) to determine when the loop stops. Taking the log on both sides gives us \(x \geq log_2(n)\). This gives us a time complexity of O(log n). \\
\end{myindentpar}
\pagebreak
\section*{Part 2: Time Complexity}
\subsection*{\underline{A. Algorithm Analysis:}}
Consider \textbf{arr} is a reference to an unsorted array, analyze the following function: 
\begin{lstlisting}
public boolean search(int[] arr, int target) 
{
  for (int i = 0; i < arr.length; i++) 
  {
    if (arr[i] == target) 
    {
      return true;
    }
  }
  return false;
}
\end{lstlisting}
What is the time complexity of the best-case and worst-case scenarios for this search?\\ \\
\textbf{Solution:}
\begin{myindentpar}{.5cm}
\textbf{Best-Case:} O(1)
\begin{myindentpar}{.5cm}
    If target is the first index then the function returns true immediately.
\end{myindentpar}
\textbf{Worst-Case:} O(n) if n is equal to the length of the array
\begin{myindentpar}{.5cm}
    If the target is not present or in the last index, then the loop runs n times.
\end{myindentpar}
\end{myindentpar}

\subsection*{\underline{B. Algorithm Comparison:}}
Three algorithms solve the same problem:
\begin{itemize}
    \item Algorithm A: \(O(n^2)\)
    \item Algorithm B: \(O(n\log n)\)
    \item Algorithm C: \(O(n!)\)
\end{itemize}
Which algorithm would you choose and why? Provide a detailed explanation comparing their performance as \(n\) grows.\\ \\
\textbf{Solution:}
\begin{myindentpar}{.5cm}
    I would Choose Algorithm B: \(O(n\log n)\). As seen from the table below, as \(n\) grows the time complexity grows the slowest with \(O(n\log n)\) so it is the most optimal out of the three algorithms. 
\end{myindentpar}
\begin{center}
    \begin{tabular}{|c||c|c|c|}
    \hline
    n&\(O(n^2)\)&\(O(n\log n)\) &\(O(n!)\) \\ [0,5ex]
    \hline
    1 & 1 & 0 & 1 \\
    2 & 4 & .06 & 2 \\
    3 & 9 & 1.4 & 6 \\ 
    4 & 16 & 2.4 & 24 \\
    5 & 25 & 3.5 & 120 \\
    \hline
    
    
    \end{tabular}
\end{center}
\pagebreak
\section*{Part 3: Space Complexity}
Consider the following function:
\begin{lstlisting}
public int[] doubleValues(int[] arr)
{
  int[] result = new int[arr.length];
  for (int i = 0; i < arr.length; i++) 
  {
    result[i] = arr[i] * 2;
  }
  return result;
}
\end{lstlisting}
\begin{itemize}
    \item [--] What is the space complexity of this function?
    \item [--] Suggest an alternative approach to reduce the space complexity if applicable.
\end{itemize}
\textbf{Solution:}
\begin{myindentpar}{.5cm}
    The space complexity of this function is \(O(n)\) where n is equal to the length of the array. This is because the function creates a new array \textit{result} the same size as \textit{arr}, so amount of memory used grows linearly every iteration. \\ \\
    You can reduce the space complexity by modifying the input array rather than creating a new array. The code for this is shown below.
    \begin{lstlisting}
public int doubleValues(int[] arr) 
{
  for (int i = 0; i < arr.length; i++) 
  {
    arr[i] *= 2;
  }
  return result;
}
    \end{lstlisting}
\end{myindentpar}

\pagebreak

\section*{Part 4: Design Algorithm}
Design an algorithm to find the second largest number in an array with the time complexity of \(O(n)\) and a space complexity of \(O(1)\). Provide your implementation and explain how it meets the time and space complexity requirements. \\ \\
\textbf{IMPLEMENTATION GUIDELINES} 
\begin{itemize}
    \item [--]Your approach must achieve a time complexity of \(O(n)\) by avoiding nested loops and ensuring each element in the array is processed only once. Nested loop approaches result in \(O(n^2)\) time complexity, which is less efficient and does not meet the requirements.
    \item [--]Your code should use a single loop statement, but there is no limit on the number of conditional statements (e.g., if).
    \item[--]Your implementation should handle arrays of any size without limitations.
    \item[--] Built-in methods such as max should not be used. Finding the largest element must be done using conditional statements to compare array elements
\end{itemize}

\noindent{\textbf{Solution :}}\\
\begin{lstlisting}
class SecondLargestFinder {
    public int findSecondLargest(int[] arr) {
        int largest = Integer.MIN_VALUE, secondLargest = Integer.MIN_VALUE;
        for (int num : arr) {
            if (num > largest) {
                secondLargest = largest;
                largest = num;
            } else if (num > secondLargest && num < largest) {
                secondLargest = num;
            }
        }
        return secondLargest;
    }
}

\end{lstlisting}

\end{document}
